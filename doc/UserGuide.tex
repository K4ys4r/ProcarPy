
% Default to the notebook output style

    


% Inherit from the specified cell style.




    
\documentclass[11pt]{article}

    
    
    \usepackage[T1]{fontenc}
    % Nicer default font (+ math font) than Computer Modern for most use cases
    \usepackage{mathpazo}

    % Basic figure setup, for now with no caption control since it's done
    % automatically by Pandoc (which extracts ![](path) syntax from Markdown).
    \usepackage{graphicx}
    % We will generate all images so they have a width \maxwidth. This means
    % that they will get their normal width if they fit onto the page, but
    % are scaled down if they would overflow the margins.
    \makeatletter
    \def\maxwidth{\ifdim\Gin@nat@width>\linewidth\linewidth
    \else\Gin@nat@width\fi}
    \makeatother
    \let\Oldincludegraphics\includegraphics
    % Set max figure width to be 80% of text width, for now hardcoded.
    \renewcommand{\includegraphics}[1]{\Oldincludegraphics[width=.8\maxwidth]{#1}}
    % Ensure that by default, figures have no caption (until we provide a
    % proper Figure object with a Caption API and a way to capture that
    % in the conversion process - todo).
    \usepackage{caption}
    \DeclareCaptionLabelFormat{nolabel}{}
    \captionsetup{labelformat=nolabel}

    \usepackage{adjustbox} % Used to constrain images to a maximum size 
    \usepackage{xcolor} % Allow colors to be defined
    \usepackage{enumerate} % Needed for markdown enumerations to work
    \usepackage{geometry} % Used to adjust the document margins
    \usepackage{amsmath} % Equations
    \usepackage{amssymb} % Equations
    \usepackage{textcomp} % defines textquotesingle
    % Hack from http://tex.stackexchange.com/a/47451/13684:
    \AtBeginDocument{%
        \def\PYZsq{\textquotesingle}% Upright quotes in Pygmentized code
    }
    \usepackage{upquote} % Upright quotes for verbatim code
    \usepackage{eurosym} % defines \euro
    \usepackage[mathletters]{ucs} % Extended unicode (utf-8) support
    \usepackage[utf8x]{inputenc} % Allow utf-8 characters in the tex document
    \usepackage{fancyvrb} % verbatim replacement that allows latex
    \usepackage{grffile} % extends the file name processing of package graphics 
                         % to support a larger range 
    % The hyperref package gives us a pdf with properly built
    % internal navigation ('pdf bookmarks' for the table of contents,
    % internal cross-reference links, web links for URLs, etc.)
    \usepackage{hyperref}
    \usepackage{longtable} % longtable support required by pandoc >1.10
    \usepackage{booktabs}  % table support for pandoc > 1.12.2
    \usepackage[inline]{enumitem} % IRkernel/repr support (it uses the enumerate* environment)
    \usepackage[normalem]{ulem} % ulem is needed to support strikethroughs (\sout)
                                % normalem makes italics be italics, not underlines
    \usepackage{mathrsfs}
    

    
    
    % Colors for the hyperref package
    \definecolor{urlcolor}{rgb}{0,.145,.698}
    \definecolor{linkcolor}{rgb}{.71,0.21,0.01}
    \definecolor{citecolor}{rgb}{.12,.54,.11}

    % ANSI colors
    \definecolor{ansi-black}{HTML}{3E424D}
    \definecolor{ansi-black-intense}{HTML}{282C36}
    \definecolor{ansi-red}{HTML}{E75C58}
    \definecolor{ansi-red-intense}{HTML}{B22B31}
    \definecolor{ansi-green}{HTML}{00A250}
    \definecolor{ansi-green-intense}{HTML}{007427}
    \definecolor{ansi-yellow}{HTML}{DDB62B}
    \definecolor{ansi-yellow-intense}{HTML}{B27D12}
    \definecolor{ansi-blue}{HTML}{208FFB}
    \definecolor{ansi-blue-intense}{HTML}{0065CA}
    \definecolor{ansi-magenta}{HTML}{D160C4}
    \definecolor{ansi-magenta-intense}{HTML}{A03196}
    \definecolor{ansi-cyan}{HTML}{60C6C8}
    \definecolor{ansi-cyan-intense}{HTML}{258F8F}
    \definecolor{ansi-white}{HTML}{C5C1B4}
    \definecolor{ansi-white-intense}{HTML}{A1A6B2}
    \definecolor{ansi-default-inverse-fg}{HTML}{FFFFFF}
    \definecolor{ansi-default-inverse-bg}{HTML}{000000}

    % commands and environments needed by pandoc snippets
    % extracted from the output of `pandoc -s`
    \providecommand{\tightlist}{%
      \setlength{\itemsep}{0pt}\setlength{\parskip}{0pt}}
    \DefineVerbatimEnvironment{Highlighting}{Verbatim}{commandchars=\\\{\}}
    % Add ',fontsize=\small' for more characters per line
    \newenvironment{Shaded}{}{}
    \newcommand{\KeywordTok}[1]{\textcolor[rgb]{0.00,0.44,0.13}{\textbf{{#1}}}}
    \newcommand{\DataTypeTok}[1]{\textcolor[rgb]{0.56,0.13,0.00}{{#1}}}
    \newcommand{\DecValTok}[1]{\textcolor[rgb]{0.25,0.63,0.44}{{#1}}}
    \newcommand{\BaseNTok}[1]{\textcolor[rgb]{0.25,0.63,0.44}{{#1}}}
    \newcommand{\FloatTok}[1]{\textcolor[rgb]{0.25,0.63,0.44}{{#1}}}
    \newcommand{\CharTok}[1]{\textcolor[rgb]{0.25,0.44,0.63}{{#1}}}
    \newcommand{\StringTok}[1]{\textcolor[rgb]{0.25,0.44,0.63}{{#1}}}
    \newcommand{\CommentTok}[1]{\textcolor[rgb]{0.38,0.63,0.69}{\textit{{#1}}}}
    \newcommand{\OtherTok}[1]{\textcolor[rgb]{0.00,0.44,0.13}{{#1}}}
    \newcommand{\AlertTok}[1]{\textcolor[rgb]{1.00,0.00,0.00}{\textbf{{#1}}}}
    \newcommand{\FunctionTok}[1]{\textcolor[rgb]{0.02,0.16,0.49}{{#1}}}
    \newcommand{\RegionMarkerTok}[1]{{#1}}
    \newcommand{\ErrorTok}[1]{\textcolor[rgb]{1.00,0.00,0.00}{\textbf{{#1}}}}
    \newcommand{\NormalTok}[1]{{#1}}
    
    % Additional commands for more recent versions of Pandoc
    \newcommand{\ConstantTok}[1]{\textcolor[rgb]{0.53,0.00,0.00}{{#1}}}
    \newcommand{\SpecialCharTok}[1]{\textcolor[rgb]{0.25,0.44,0.63}{{#1}}}
    \newcommand{\VerbatimStringTok}[1]{\textcolor[rgb]{0.25,0.44,0.63}{{#1}}}
    \newcommand{\SpecialStringTok}[1]{\textcolor[rgb]{0.73,0.40,0.53}{{#1}}}
    \newcommand{\ImportTok}[1]{{#1}}
    \newcommand{\DocumentationTok}[1]{\textcolor[rgb]{0.73,0.13,0.13}{\textit{{#1}}}}
    \newcommand{\AnnotationTok}[1]{\textcolor[rgb]{0.38,0.63,0.69}{\textbf{\textit{{#1}}}}}
    \newcommand{\CommentVarTok}[1]{\textcolor[rgb]{0.38,0.63,0.69}{\textbf{\textit{{#1}}}}}
    \newcommand{\VariableTok}[1]{\textcolor[rgb]{0.10,0.09,0.49}{{#1}}}
    \newcommand{\ControlFlowTok}[1]{\textcolor[rgb]{0.00,0.44,0.13}{\textbf{{#1}}}}
    \newcommand{\OperatorTok}[1]{\textcolor[rgb]{0.40,0.40,0.40}{{#1}}}
    \newcommand{\BuiltInTok}[1]{{#1}}
    \newcommand{\ExtensionTok}[1]{{#1}}
    \newcommand{\PreprocessorTok}[1]{\textcolor[rgb]{0.74,0.48,0.00}{{#1}}}
    \newcommand{\AttributeTok}[1]{\textcolor[rgb]{0.49,0.56,0.16}{{#1}}}
    \newcommand{\InformationTok}[1]{\textcolor[rgb]{0.38,0.63,0.69}{\textbf{\textit{{#1}}}}}
    \newcommand{\WarningTok}[1]{\textcolor[rgb]{0.38,0.63,0.69}{\textbf{\textit{{#1}}}}}
    
    
    % Define a nice break command that doesn't care if a line doesn't already
    % exist.
    \def\br{\hspace*{\fill} \\* }
    % Math Jax compatibility definitions
    \def\gt{>}
    \def\lt{<}
    \let\Oldtex\TeX
    \let\Oldlatex\LaTeX
    \renewcommand{\TeX}{\textrm{\Oldtex}}
    \renewcommand{\LaTeX}{\textrm{\Oldlatex}}
    % Document parameters
    % Document title
    \title{ProcarPy UserGuide}
    
    
    
    
    

    % Pygments definitions
    
\makeatletter
\def\PY@reset{\let\PY@it=\relax \let\PY@bf=\relax%
    \let\PY@ul=\relax \let\PY@tc=\relax%
    \let\PY@bc=\relax \let\PY@ff=\relax}
\def\PY@tok#1{\csname PY@tok@#1\endcsname}
\def\PY@toks#1+{\ifx\relax#1\empty\else%
    \PY@tok{#1}\expandafter\PY@toks\fi}
\def\PY@do#1{\PY@bc{\PY@tc{\PY@ul{%
    \PY@it{\PY@bf{\PY@ff{#1}}}}}}}
\def\PY#1#2{\PY@reset\PY@toks#1+\relax+\PY@do{#2}}

\expandafter\def\csname PY@tok@kd\endcsname{\let\PY@bf=\textbf\def\PY@tc##1{\textcolor[rgb]{0.00,0.50,0.00}{##1}}}
\expandafter\def\csname PY@tok@m\endcsname{\def\PY@tc##1{\textcolor[rgb]{0.40,0.40,0.40}{##1}}}
\expandafter\def\csname PY@tok@sh\endcsname{\def\PY@tc##1{\textcolor[rgb]{0.73,0.13,0.13}{##1}}}
\expandafter\def\csname PY@tok@na\endcsname{\def\PY@tc##1{\textcolor[rgb]{0.49,0.56,0.16}{##1}}}
\expandafter\def\csname PY@tok@w\endcsname{\def\PY@tc##1{\textcolor[rgb]{0.73,0.73,0.73}{##1}}}
\expandafter\def\csname PY@tok@s2\endcsname{\def\PY@tc##1{\textcolor[rgb]{0.73,0.13,0.13}{##1}}}
\expandafter\def\csname PY@tok@ge\endcsname{\let\PY@it=\textit}
\expandafter\def\csname PY@tok@sa\endcsname{\def\PY@tc##1{\textcolor[rgb]{0.73,0.13,0.13}{##1}}}
\expandafter\def\csname PY@tok@kc\endcsname{\let\PY@bf=\textbf\def\PY@tc##1{\textcolor[rgb]{0.00,0.50,0.00}{##1}}}
\expandafter\def\csname PY@tok@nb\endcsname{\def\PY@tc##1{\textcolor[rgb]{0.00,0.50,0.00}{##1}}}
\expandafter\def\csname PY@tok@mh\endcsname{\def\PY@tc##1{\textcolor[rgb]{0.40,0.40,0.40}{##1}}}
\expandafter\def\csname PY@tok@kn\endcsname{\let\PY@bf=\textbf\def\PY@tc##1{\textcolor[rgb]{0.00,0.50,0.00}{##1}}}
\expandafter\def\csname PY@tok@mb\endcsname{\def\PY@tc##1{\textcolor[rgb]{0.40,0.40,0.40}{##1}}}
\expandafter\def\csname PY@tok@nv\endcsname{\def\PY@tc##1{\textcolor[rgb]{0.10,0.09,0.49}{##1}}}
\expandafter\def\csname PY@tok@cm\endcsname{\let\PY@it=\textit\def\PY@tc##1{\textcolor[rgb]{0.25,0.50,0.50}{##1}}}
\expandafter\def\csname PY@tok@gr\endcsname{\def\PY@tc##1{\textcolor[rgb]{1.00,0.00,0.00}{##1}}}
\expandafter\def\csname PY@tok@gs\endcsname{\let\PY@bf=\textbf}
\expandafter\def\csname PY@tok@vi\endcsname{\def\PY@tc##1{\textcolor[rgb]{0.10,0.09,0.49}{##1}}}
\expandafter\def\csname PY@tok@nf\endcsname{\def\PY@tc##1{\textcolor[rgb]{0.00,0.00,1.00}{##1}}}
\expandafter\def\csname PY@tok@mi\endcsname{\def\PY@tc##1{\textcolor[rgb]{0.40,0.40,0.40}{##1}}}
\expandafter\def\csname PY@tok@bp\endcsname{\def\PY@tc##1{\textcolor[rgb]{0.00,0.50,0.00}{##1}}}
\expandafter\def\csname PY@tok@gd\endcsname{\def\PY@tc##1{\textcolor[rgb]{0.63,0.00,0.00}{##1}}}
\expandafter\def\csname PY@tok@se\endcsname{\let\PY@bf=\textbf\def\PY@tc##1{\textcolor[rgb]{0.73,0.40,0.13}{##1}}}
\expandafter\def\csname PY@tok@ss\endcsname{\def\PY@tc##1{\textcolor[rgb]{0.10,0.09,0.49}{##1}}}
\expandafter\def\csname PY@tok@mf\endcsname{\def\PY@tc##1{\textcolor[rgb]{0.40,0.40,0.40}{##1}}}
\expandafter\def\csname PY@tok@sr\endcsname{\def\PY@tc##1{\textcolor[rgb]{0.73,0.40,0.53}{##1}}}
\expandafter\def\csname PY@tok@s1\endcsname{\def\PY@tc##1{\textcolor[rgb]{0.73,0.13,0.13}{##1}}}
\expandafter\def\csname PY@tok@vm\endcsname{\def\PY@tc##1{\textcolor[rgb]{0.10,0.09,0.49}{##1}}}
\expandafter\def\csname PY@tok@vg\endcsname{\def\PY@tc##1{\textcolor[rgb]{0.10,0.09,0.49}{##1}}}
\expandafter\def\csname PY@tok@kr\endcsname{\let\PY@bf=\textbf\def\PY@tc##1{\textcolor[rgb]{0.00,0.50,0.00}{##1}}}
\expandafter\def\csname PY@tok@s\endcsname{\def\PY@tc##1{\textcolor[rgb]{0.73,0.13,0.13}{##1}}}
\expandafter\def\csname PY@tok@sc\endcsname{\def\PY@tc##1{\textcolor[rgb]{0.73,0.13,0.13}{##1}}}
\expandafter\def\csname PY@tok@ch\endcsname{\let\PY@it=\textit\def\PY@tc##1{\textcolor[rgb]{0.25,0.50,0.50}{##1}}}
\expandafter\def\csname PY@tok@ow\endcsname{\let\PY@bf=\textbf\def\PY@tc##1{\textcolor[rgb]{0.67,0.13,1.00}{##1}}}
\expandafter\def\csname PY@tok@sd\endcsname{\let\PY@it=\textit\def\PY@tc##1{\textcolor[rgb]{0.73,0.13,0.13}{##1}}}
\expandafter\def\csname PY@tok@mo\endcsname{\def\PY@tc##1{\textcolor[rgb]{0.40,0.40,0.40}{##1}}}
\expandafter\def\csname PY@tok@k\endcsname{\let\PY@bf=\textbf\def\PY@tc##1{\textcolor[rgb]{0.00,0.50,0.00}{##1}}}
\expandafter\def\csname PY@tok@dl\endcsname{\def\PY@tc##1{\textcolor[rgb]{0.73,0.13,0.13}{##1}}}
\expandafter\def\csname PY@tok@cp\endcsname{\def\PY@tc##1{\textcolor[rgb]{0.74,0.48,0.00}{##1}}}
\expandafter\def\csname PY@tok@cpf\endcsname{\let\PY@it=\textit\def\PY@tc##1{\textcolor[rgb]{0.25,0.50,0.50}{##1}}}
\expandafter\def\csname PY@tok@nl\endcsname{\def\PY@tc##1{\textcolor[rgb]{0.63,0.63,0.00}{##1}}}
\expandafter\def\csname PY@tok@sb\endcsname{\def\PY@tc##1{\textcolor[rgb]{0.73,0.13,0.13}{##1}}}
\expandafter\def\csname PY@tok@c1\endcsname{\let\PY@it=\textit\def\PY@tc##1{\textcolor[rgb]{0.25,0.50,0.50}{##1}}}
\expandafter\def\csname PY@tok@o\endcsname{\def\PY@tc##1{\textcolor[rgb]{0.40,0.40,0.40}{##1}}}
\expandafter\def\csname PY@tok@nt\endcsname{\let\PY@bf=\textbf\def\PY@tc##1{\textcolor[rgb]{0.00,0.50,0.00}{##1}}}
\expandafter\def\csname PY@tok@gu\endcsname{\let\PY@bf=\textbf\def\PY@tc##1{\textcolor[rgb]{0.50,0.00,0.50}{##1}}}
\expandafter\def\csname PY@tok@err\endcsname{\def\PY@bc##1{\setlength{\fboxsep}{0pt}\fcolorbox[rgb]{1.00,0.00,0.00}{1,1,1}{\strut ##1}}}
\expandafter\def\csname PY@tok@nc\endcsname{\let\PY@bf=\textbf\def\PY@tc##1{\textcolor[rgb]{0.00,0.00,1.00}{##1}}}
\expandafter\def\csname PY@tok@ne\endcsname{\let\PY@bf=\textbf\def\PY@tc##1{\textcolor[rgb]{0.82,0.25,0.23}{##1}}}
\expandafter\def\csname PY@tok@vc\endcsname{\def\PY@tc##1{\textcolor[rgb]{0.10,0.09,0.49}{##1}}}
\expandafter\def\csname PY@tok@kp\endcsname{\def\PY@tc##1{\textcolor[rgb]{0.00,0.50,0.00}{##1}}}
\expandafter\def\csname PY@tok@nn\endcsname{\let\PY@bf=\textbf\def\PY@tc##1{\textcolor[rgb]{0.00,0.00,1.00}{##1}}}
\expandafter\def\csname PY@tok@ni\endcsname{\let\PY@bf=\textbf\def\PY@tc##1{\textcolor[rgb]{0.60,0.60,0.60}{##1}}}
\expandafter\def\csname PY@tok@si\endcsname{\let\PY@bf=\textbf\def\PY@tc##1{\textcolor[rgb]{0.73,0.40,0.53}{##1}}}
\expandafter\def\csname PY@tok@kt\endcsname{\def\PY@tc##1{\textcolor[rgb]{0.69,0.00,0.25}{##1}}}
\expandafter\def\csname PY@tok@il\endcsname{\def\PY@tc##1{\textcolor[rgb]{0.40,0.40,0.40}{##1}}}
\expandafter\def\csname PY@tok@gi\endcsname{\def\PY@tc##1{\textcolor[rgb]{0.00,0.63,0.00}{##1}}}
\expandafter\def\csname PY@tok@gp\endcsname{\let\PY@bf=\textbf\def\PY@tc##1{\textcolor[rgb]{0.00,0.00,0.50}{##1}}}
\expandafter\def\csname PY@tok@nd\endcsname{\def\PY@tc##1{\textcolor[rgb]{0.67,0.13,1.00}{##1}}}
\expandafter\def\csname PY@tok@no\endcsname{\def\PY@tc##1{\textcolor[rgb]{0.53,0.00,0.00}{##1}}}
\expandafter\def\csname PY@tok@go\endcsname{\def\PY@tc##1{\textcolor[rgb]{0.53,0.53,0.53}{##1}}}
\expandafter\def\csname PY@tok@fm\endcsname{\def\PY@tc##1{\textcolor[rgb]{0.00,0.00,1.00}{##1}}}
\expandafter\def\csname PY@tok@sx\endcsname{\def\PY@tc##1{\textcolor[rgb]{0.00,0.50,0.00}{##1}}}
\expandafter\def\csname PY@tok@cs\endcsname{\let\PY@it=\textit\def\PY@tc##1{\textcolor[rgb]{0.25,0.50,0.50}{##1}}}
\expandafter\def\csname PY@tok@c\endcsname{\let\PY@it=\textit\def\PY@tc##1{\textcolor[rgb]{0.25,0.50,0.50}{##1}}}
\expandafter\def\csname PY@tok@gt\endcsname{\def\PY@tc##1{\textcolor[rgb]{0.00,0.27,0.87}{##1}}}
\expandafter\def\csname PY@tok@gh\endcsname{\let\PY@bf=\textbf\def\PY@tc##1{\textcolor[rgb]{0.00,0.00,0.50}{##1}}}

\def\PYZbs{\char`\\}
\def\PYZus{\char`\_}
\def\PYZob{\char`\{}
\def\PYZcb{\char`\}}
\def\PYZca{\char`\^}
\def\PYZam{\char`\&}
\def\PYZlt{\char`\<}
\def\PYZgt{\char`\>}
\def\PYZsh{\char`\#}
\def\PYZpc{\char`\%}
\def\PYZdl{\char`\$}
\def\PYZhy{\char`\-}
\def\PYZsq{\char`\'}
\def\PYZdq{\char`\"}
\def\PYZti{\char`\~}
% for compatibility with earlier versions
\def\PYZat{@}
\def\PYZlb{[}
\def\PYZrb{]}
\makeatother


    % Exact colors from NB
    \definecolor{incolor}{rgb}{0.0, 0.0, 0.5}
    \definecolor{outcolor}{rgb}{0.545, 0.0, 0.0}



    
    % Prevent overflowing lines due to hard-to-break entities
    \sloppy 
    % Setup hyperref package
    \hypersetup{
      breaklinks=true,  % so long urls are correctly broken across lines
      colorlinks=true,
      urlcolor=urlcolor,
      linkcolor=linkcolor,
      citecolor=citecolor,
      }
    % Slightly bigger margins than the latex defaults
    
    \geometry{verbose,tmargin=1in,bmargin=1in,lmargin=1in,rmargin=1in}
    
\author{Hilal Balout}    

    \begin{document}
    
    
    \maketitle
    
    

    
    \section{ProcarPy}\label{procarpy}

\subsection{What is ProcarPy}\label{what-is-procarpy}

\textbf{ProcarPy} is a \href{https://www.python.org/}{\textbf{Python}}
module which allows to parse and plot the electronic band structure
diagram from \href{https://www.vasp.at/}{\textbf{VASP}} PROCAR file.
\href{https://www.numpy.org/}{\textbf{Numpy}} and
\href{https://matplotlib.org/}{\textbf{Matplotlib}} packages are
required.

\subsection{What ProcarPy can do ?}\label{what-procarpy-can-do}

There is different
\href{https://cms.mpi.univie.ac.at/wiki/index.php/PROCAR}{\textbf{PROCAR
output format}}. However ProcarPy can parse and process all PROCAR files
from tags below:

\begin{itemize}
\tightlist
\item
  LORBIT = 10 : PROCAR not decomposed (\(s\), \(p\), \(d\)).
\item
  LORBIT = 11 : PROCAR lm decomposed (\(s\), \(p_y\), \(p_z\), \(p_x\),
  \(d_{xy}\), \(d_{yz}\), ...).
\item
  ISPIN = 2 : Spin Polarized Calculation (\(s_{up}\), \(s_{down}\),
  \(p_{up}\), \(p_{down}\), ...).
\item
  LNONCOLLINEAR=.TRUE. : Spin-orbit coupling.
\end{itemize}

Then, with ProcarPy, the total and projected electronic band structure
can be plotted.

\subsection{ProcarPy Class \& Methods}\label{procarpy-class-methods}

\subsubsection{PROCARBandStructure
Class}\label{procarbandstructure-class}

It is the principale calss of ProcarPy module. It takes many parameters:

\begin{Shaded}
\begin{Highlighting}[]
\KeywordTok{class} \NormalTok{PROCARBandStructure:}
    \KeywordTok{def} \FunctionTok{__init__}\NormalTok{(}\VariableTok{self}\NormalTok{, filename, path, ef, pathstyle, SO, spin):}
\end{Highlighting}
\end{Shaded}

All parameters are described below:

\begin{itemize}
\tightlist
\item
  \textbf{filename :} is a \textbf{\emph{string}} corresponds the PROCAR
  file name.
\item
  \textbf{path :} is a \textbf{\emph{list}} of strings includes the
  \(k-\)points labels used in bands calculation (default value :
  {[}{]}).

  \begin{itemize}
  \tightlist
  \item
    i.e : {[}"\(L\)","\(\Gamma\)","\(X\)"{]}.
  \end{itemize}
\item
  \textbf{ef :} is a \textbf{\emph{float}} corresponds to the Fermi
  energy in \emph{eV} (default value : None).
\item
  \textbf{pathstyle :} is a \textbf{\emph{string}} corresponds the
  nature of k-points path if it is continuous or discontinuous (default
  value : discontinuous).

  \begin{itemize}
  \tightlist
  \item
    \emph{pathstyle = "discontinuous"} : the k-points absciss will be a
    sequence of numbers from 0 to k-points numbers.
  \item
    \emph{pathstyle = "continuous"} : the k-points absciss will be
    calculated as below:
  \end{itemize}
\end{itemize}

\begin{equation*}
k_{absciss}[i] = k_{absciss} [i-1] + \mid\mid \vec{k[i]} \mid\mid ~; ~ with ~ k_{absciss}[0] = 0
\end{equation*}

\begin{itemize}
\tightlist
\item
  \textbf{SO :} is a \textbf{\emph{boolean}} indicates if Spin-orbit
  coupling was used or not (default value : False).
\item
  \textbf{spin :} is a \textbf{\emph{boolean}} indicates if
  Spin-polarized calculations were performed or not (default value :
  False).
\end{itemize}

\subsubsection{Init\_Fig Method}\label{initux5ffig-method}

This method initializes the plot figure . It takes two parameters.

\begin{Shaded}
\begin{Highlighting}[]
\KeywordTok{def} \NormalTok{Init_Fig(}\VariableTok{self}\NormalTok{, width, height):}
\end{Highlighting}
\end{Shaded}

\begin{itemize}
\tightlist
\item
  \textbf{width :} is a \textbf{\emph{float}} defines the figure width
  in inch (default value : 8.0 inch).
\item
  \textbf{height :} is a \textbf{\emph{float}} defines the figure heigth
  in inch (default value : 6.0 inch).
\end{itemize}

\subsubsection{getTotalBandsPlot Method}\label{gettotalbandsplot-method}

This method allows to plot the total electronic band structure from
PROCAR file. Its input parameters are described below:

\begin{Shaded}
\begin{Highlighting}[]
\KeywordTok{def} \NormalTok{getTotalBandsPlot(}\VariableTok{self}\NormalTok{, bandspin, lw, color, alpha, label):}
\end{Highlighting}
\end{Shaded}

\begin{itemize}
\tightlist
\item
  \textbf{bandspin :} sets the bands spin to plot:

  \begin{itemize}
  \tightlist
  \item
    if spin = True : bandspin can be "up" or "down".
  \item
    else : bandspin is not taken into account.
  \end{itemize}
\item
  \textbf{lw :} sets the line width in points (default value : 1).
\item
  \textbf{color :} stes the line colors (default value : blue).
\item
  \textbf{alpha :} sets the transparency of color (default value : 1).
\item
  \textbf{label :} sets the plot label for auto legend (default value :
  Total Band and/or Spin).
\end{itemize}

\subsubsection{getOrbitalBandsPlot
Method}\label{getorbitalbandsplot-method}

This method allows to plot the Projected Band structure of atomic
orbital (\(s\), \(p\), \(p_x\), \(p_y\), ...) for a defined atom. Its
input parameters are described below:

\begin{Shaded}
\begin{Highlighting}[]
\KeywordTok{def} \NormalTok{getOrbitalBandsPlot(}\VariableTok{self}\NormalTok{,orbital, atom, magn, sign, marker, color, alpha, scale, label):}
\end{Highlighting}
\end{Shaded}

\begin{itemize}
\item
  \textbf{orbital :} sets the atomic orbital.

  \begin{itemize}
  \tightlist
  \item
    if spin = True (default Value "tot\_up"):

    \begin{itemize}
    \item
      LORBIT = 10: orbital can be one of the strings below:

\begin{verbatim}
    "s_up"   ,   "s_down"  ,
    "p_up"   ,   "p_down"  ,
    "d_up"   ,   "d_down"  ,
    "tot_up" ,   "tot_down".
\end{verbatim}
    \item
      LORBIT = 11: orbital can be one of the strings below:

\begin{verbatim}
    "s_up"      ,   "s_down"      ,     
    "px_up"     ,   "px_down"     ,     
    "py_up"     ,   "py_down"     ,     
    "pz_up"     ,   "pz_down"     ,    
    "dxy_up"    ,   "dxy_down"    ,    
    "dyz_up"    ,   "dyz_down"    ,    
    "dz2_up"    ,   "dz2_down"    ,    
    "dxz_up"    ,   "dxz_down"    ,    
    "dx2-y2_up" ,   "dx2-y2_down" ,
    "tot_up"    ,   "tot_down".
\end{verbatim}
    \end{itemize}
  \item
    if spin = False (default Value "tot"):

    \begin{itemize}
    \item
      LORBIT = 10: orbital can be one of the strings below:

\begin{verbatim}
    "s" , "p" , "d", "tot"
\end{verbatim}
    \item
      LORBIT = 11: orbital can be one of the strings below:

\begin{verbatim}
    "s"  ,
    "px" , "py" , "pz" ,
    "dxy", "dyz", "dz2", "dxz", "dx2-y2",
    "tot" .
\end{verbatim}
    \end{itemize}
  \end{itemize}
\item
  \textbf{atom :} sets the atom index (integer, default value = 1).
\item
  \textbf{magn :} sets the projected magnetizations.

  \begin{itemize}
  \item
    if SO = True (default Value "mtot"): magn can be one of the strings
    below:

\begin{verbatim}
        "mtot", "mx", "my", "mz".
\end{verbatim}
  \item
    else: magn is not taken into account.
  \end{itemize}
\item
  \textbf{sign :} sets the orbitales contribution sign for spin-orbite
  coupling calculations.

  \begin{itemize}
  \tightlist
  \item
    if SO = True and magn != mtot : sign can be "+" or "-".
  \item
    else: sign is not taken into account.
  \end{itemize}
\item
  \textbf{marker :} sets the plot marker. All possible markers are
  defined \href{https://matplotlib.org/api/markers_api.html}{here}.
\item
  \textbf{color :} sets the marker colors (default value : blue).
\item
  \textbf{alpha :} sets the transparency of color (default value : 1).
\item
  \textbf{scale :} sets the size of marker. The the projections value
  for every atom, in PROCAR file, are between 0 and 1. To be
  representative the default value of scale is setted to 100.
\item
  \textbf{label :} sets the plot label for auto legend (default value :
  Atom number + orbital).
\end{itemize}

\subsubsection{getAtomsRangeBandsPlot}\label{getatomsrangebandsplot}

This method allows to plot the Projected Band structure of atomic
orbital (\(s\), \(p\), \(p_x\), \(p_y\), ...) for a defined range of
atoms. Its input parameters are described below:

\begin{Shaded}
\begin{Highlighting}[]
\KeywordTok{def} \NormalTok{getAtomsRangeBandsPlot(}\VariableTok{self}\NormalTok{, orbital, AtomRange, magn, sign, marker, color, alpha, scale, label):}
\end{Highlighting}
\end{Shaded}

\begin{itemize}
\item
  \textbf{orbital :} as described above for getOrbitalBandsPlot method.
\item
  \textbf{AtomRange :} sets a range of atoms index:

  \begin{itemize}
  \tightlist
  \item
    if AtomRange is given (list of numbers) i.e {[}1, 2, 5, 10{]} to
    plot information of atoms 1, 2, 5 and 10.
  \item
    else the default values are a sequence of numbers from 0 to the
    atoms number.
  \end{itemize}
\item
  \textbf{magn :} as described above for getOrbitalBandsPlot method.
\item
  \textbf{sign :} as described above for getOrbitalBandsPlot method.
\item
  \textbf{marker :} as described above for getOrbitalBandsPlot method.
\item
  \textbf{color :} as described above for getOrbitalBandsPlot method.
\item
  \textbf{alpha :} as described above for getOrbitalBandsPlot method.
\item
  \textbf{scale :} as described above for getOrbitalBandsPlot method.
\item
  \textbf{label :} as described above for getOrbitalBandsPlot method.
\end{itemize}

\subsubsection{getBandsGap Method}\label{getbandsgap-method}

This method allows to calculate the band gap between two defined bands.
It takes three parameters:

\begin{Shaded}
\begin{Highlighting}[]
\KeywordTok{def} \NormalTok{getBandsGap(}\VariableTok{self}\NormalTok{,band1,band2,bandspin):}
\end{Highlighting}
\end{Shaded}

\begin{itemize}
\tightlist
\item
  \textbf{band1 :} sets the first band index.
\item
  \textbf{band2 :} sets the second band index.
\item
  \textbf{bandspin :} sets the band spin ("up", "down") if spin = True.
\end{itemize}

\subsubsection{PlotShow Method}\label{plotshow-method}

This method allows to show or save the band structures plot.

\begin{Shaded}
\begin{Highlighting}[]
\KeywordTok{def} \NormalTok{PlotShow(}\VariableTok{self}\NormalTok{, ymin, ymax, xmin, xmax, savefile):}
\end{Highlighting}
\end{Shaded}

\begin{itemize}
\tightlist
\item
  \textbf{savefile :} sets the output file name. i.e "image1.png",
  "output.pdf" (default value : showing of the band structures plot).
\item
  \textbf{xmin :} sets the minimum value of x-axis (default value : axis
  xmin).
\item
  \textbf{xmax :} sets the maximum value of x-axis (default value : axis
  xmax).
\item
  \textbf{ymin :} sets the minimum value of y-axis (default value : axis
  ymin).
\item
  \textbf{ymax :} sets the maximum value of y-axis (default value : axis
  ymax).
\end{itemize}

\subsection{Installation}\label{installation}

ProcarPy source can be downloaded from
\href{https://github.com/K4ys4r/ProcarPy}{GitHub Repository}. Or by git
command:

\begin{verbatim}
git clone https://github.com/K4ys4r/ProcarPy
\end{verbatim}

Change (cd) to the ProcarPy directory, and then, on the command line,
enter:

\begin{verbatim}
Python setup.py install
\end{verbatim}

\subsection{Report Bugs}\label{report-bugs}

Report bugs at \href{https://github.com/K4ys4r/ProcarPy/issues}{Bugs}.
For every bub, it is therfore appropriate to specify the:

\begin{itemize}
\tightlist
\item
  Operating system name and version.
\item
  Python version.
\item
  Detailed steps to reproduce the bug.
\end{itemize}

\subsection{Usage of ProcarPy}\label{usage-of-procarpy}

Some tutorials have been performed to show how ProcarPy can be handled.
All VASP calculations are without high precision. They are done just for
these tutorials. VASP\_5.4.4 version has been used.

\begin{itemize}
\tightlist
\item
  \href{https://github.com/K4ys4r/ProcarPy/blob/master/test/Si_diamond/Si_Diamond_Tutorial.ipynb}{\textbf{Si
  Diamond Structure}}
\item
  \href{https://github.com/K4ys4r/ProcarPy/blob/master/test/Fe2O3/Fe2O3_Tutorial.ipynb}{\textbf{Hematite
  (\(\alpha-Fe_2O3\)) Spin-Polarized calculation}}
\item
  \href{https://github.com/K4ys4r/ProcarPy/blob/master/test/Co/Co_Tutorial.ipynb}{\textbf{Cobalt
  Spin-Orbit coupling}}
\end{itemize}


    % Add a bibliography block to the postdoc
    
    
    
    \end{document}
